\documentclass[12pt]{article}%{amsart}
\usepackage[top = 1.0in, bottom = 1.0in, left = 1.0in, right = 1.0in]{geometry}
% See geometry.pdf to learn the layout options. There are lots.
%\geometry{letterpaper}               % ... or a4paper or a5paper or ...
%\geometry{landscape}             % Activate for for rotated page geometry
\usepackage[parfill]{parskip}    % Activate to begin paragraphs with an empty line rather than an indent
\usepackage{graphicx}
\usepackage{amssymb}
%\usepackage{epstopdf}
% \usepackage{caption}
\usepackage{amsmath}
% \usepackage{longtable}
% \usepackage{tabu}
% \usepackage{accents} %to get undertildes for vec & mat
\usepackage{color} % for colored text
% \usepackage[normalem]{ulem} % for editing: striking out text using 'sout' command

% \newcommand\mytablefigwidth{0.35\textwidth}

% allows to use .ai files directly w/o resaving as pdf
\DeclareGraphicsRule{.ai}{pdf}{.ai}{}

% Handy math macros!
\newcommand{\tderiv}[1]{\frac{d}{dt}#1}%{\dot{#1}}
\newcommand{\vect}[1]{\vec{#1}}
\newcommand{\rowones}{\vect{1}^\dagger}
\newcommand{\matr}[1]{\mathbf{#1}}
\newcommand{\moment}[1]{\langle #1 \rangle}
\newcommand{\vectmoment}[2]{\langle \vect{#1}_{(#2)} \rangle}
\newcommand{\conc}[1]{c_{#1}}
\newcommand{\rate}[3]{{#1}_{#2}^{#3}}
\newcommand{\mmnote}[1]{\textcolor{cyan}{(MM:~#1)}}

%%%%% Referencing macros %%%%%
\newcommand{\fref}[1]{Figure~\ref{#1}}
\newcommand{\tref}[1]{Table~\ref{#1}}
\newcommand{\eref}[1]{Eq.~(\ref{#1})}
\newcommand{\erngref}[2]{Eq.~(\ref{#1}-\ref{#2})} % Equations



%%%%%
\begin{document}

\title{Notes on theoretical model building for \textit{Xap}}

\maketitle


\section{A first pass at simple activation model}

The seemingly obvious starter model for our simple activation system is a model with four states, i.e.,
empty promoter, RNAP bound promoter, activator bound promoter, and finally RNAP and activator both bound.
An equilibrium version of this was proposed in~\cite{Bintu2005b},
but if we want to consider observables beyond mean expression levels,
we need to consider a kinetic/master equation model.
The most general version of this picture is shown in \fref{fig:4state}.
(Our attention for now is on promoter states,
ignoring mRNA production, decay, and MWC parameters for XapR.)
\begin{figure}
    \begin{center}
        \includegraphics[width=0.55\textwidth]{./simple_act.ai}
    \end{center}
    \caption{
            Promoter states for simple activation, with no assumptions made about rate constants,
            in particular to and from the doubly bound state versus the singly bound states.
            }
    \label{fig:4state}
\end{figure}
Even before considering transcription initiation rates, mRNA degradation rates,
or MWC parameters for XapR, which will collectively add at least five more parameters,
\fref{fig:4state} has eight model parameters.
Such a large number will prove difficult, at best, to infer from experiments.
This parameter proliferation highlights a serious difficulty of such a master equation model,
relative to the more economical equilibrium models.

We would like to make some assumptions to simplify this situation,
for which we propose the following:
\begin{enumerate}
    \item Set association rates equal for the singly and doubly bound states,
        i.e., assume $\rate{k}{P}{+} = \rate{\tilde{k}}{P}{+}$ and
        $\rate{k}{R}{+} = \rate{\tilde{k}}{R}{+}$.
        If association rates for XapR and RNAP are close to the diffusion limit,
        as some evidence indicates for other transcription factors (\mmnote{need cite!}),
        this implies that they bind to their specific target with high probability,
        once they are in its vicinity.
        In other words, most of their mean search time (which is 1/(on rate))
        is spent not in close proximity to their target.
        So the presence or absence of another molecular player bound at their target site should have little influence;
        either way they bind quickly, with high probability, upon arrival.
    \item If XapR and RNAP bind and unbind reversibly,
        meaning no energy is consumed and no entropy produced,
        then the cycle condition allows us to eliminate one rate
        (by expressing it as a simple algebraic function of the others).
        This condition holds even when (irreversible) transcription initiation is
        reintroduced.\footnote{This is an example of a more general result \mmnote{cite Hill's little book}.
        In an arbitrarily complicated diagram at a nonequilibrium steady-state,
        the product of forward rates around any closed path divided by reverse rates
        equals exp(free energy change around that path).}
        Note that which algebraic choices lead to the ``best'' or ``simplest'' model is not always obvious.
    \item If there is no detectable gene expression in the absence of XapR,
        as Griffin's preliminary experiments suggest,
        then we would like to excise the singly-bound RNAP state from the cartoon.
        Then there are two possible ways to proceed.
        One is to link the empty and doubly bound states with two effective rates (one each direction).
        If there is some separation of timescales between RNAP and XapR rates,
        the error from this approximation should be negligible.
        \label{item:excise_and_patch_p}
    \item The second and more drastic option, if no expression is detectable in absence of XapR,
        is to excise the RNAP bound state \textit{and} the rates leading to and from it.
        This reduces the cartoon to a three-state chain, topologically the same as our usual Lac cartoon,
        but with the roles of empty and transcription factor bound states reversed.
        This reversal is critical: some experiments that were done to infer parameters in Lac
        become either more difficult or impossible.
        This model has the same structure and same number of parameters to be fit as in Lac,
        even when we incorporate MWC model for XapR plus mRNA initiation and degradation rates.
        But again, the role reveral of empty and transcription factor bound promoter
        and the reversal of the transcription factor allosteric $\Delta\epsilon$
        will present extra experimental challenges.
    \item If on the other hand gene expression \textit{does} prove detectable in the absence of XapR,
        we cannot remove the RNAP bound state.
        This situation would reduce the scope for theory to simplify the cartoon,
        but as compensation it would make possible additional experiments
        which could independently determine some model parameters.
    \item The XapR rates in \fref{fig:4state} are implicitly only for the allosterically active state.
        As with LacI, it seems reasonable to assume that the other allosteric state
        does not bind specifically to its target binding site
        (it may bind DNA, but its binding to the target site is no stronger than to background DNA).
\end{enumerate}

The above list is presented as a buffet,
from which we may end up sampling all, some, or none of the options.

\subsection{Replacing RNAP singly bound state with effective rates}
Let us explore option \ref{item:excise_and_patch_p} in more detail.
First we point out that even for LacUV5, a ``strong'' promoter,
the weak promoter approximation has proven excellent.
In equilibrium, weak promoter means $P/K_P \ll 1$, or put differently,
the occupation probability of the RNAP bound state is small compared to the empty state.
It is hard to imagine the Xap promoter being stronger than LacUV5,
and given the low expression seen so far, it could be much weaker.
Assuming the activator XapR activates as strongly as it appears,
the doubly bound state should also have a much higher occupation probability.
Furthermore, we should expect to find a timescale separation
between the RNAP rates and the XapR rates.
If both (second order) on rates are on the order of the diffusion limit,
but the copy number of XapR is small compared to RNAP,
then the (first order) RNAP on rate should be larger than that for XapR by the same ratio.
Also, RNAP's binding to its specific sites is generally weaker than
bacterial transcription factors' specific site binding,
due to the far larger number of sites RNAP must recognize.
Essentially then, the singly bound RNAP state adds nothing but baggage to the model.
The timescale separation allows us to replace the singly bound RNAP site
with effective rates directly connecting the empty and doubly bound states,
as depicted in 
\begin{figure}
    \begin{center}
        \includegraphics[width=0.55\textwidth]{./simple_act_excisedP.ai}
    \end{center}
    \caption{
            Promoter states for simple activation,
            with RNAP singly bound state removed in favor of effective rates
            directly connecting empty and doubly-bound promoter states.
            }
    \label{fig:3state_loop}
\end{figure}

What should be the effective rates? 
First consider unbinding from the doubly bound state.
If $\rate{k}{P}{-} \gg \rate{\tilde{k}}{R}{-}, \rate{\tilde{k}}{R}{+}$ as expected,
then when XapR unbinds from the doubly bound state,
it will be followed with high probability by RNAP unbinding.
And the waiting time for RNAP unbinding will be short compared to the time
that was \textit{already} waited in the doubly bound state for XapR to unbind.

More formally, we can compute the total mean waiting time as
\begin{align}
    \frac{1}{\rate{k}{eff}{-}}
    = \frac{1}{\rate{k}{P}{-}} + \frac{1}{\rate{\tilde{k}}{R}{-}}
    = \frac{\rate{\tilde{k}}{R}{-} + \rate{k}{P}{-}}
           {\rate{\tilde{k}}{R}{-}   \rate{k}{P}{-}}
    = \frac{1}{\rate{\tilde{k}}{R}{-}}
        \left(
        \left( 1 + \frac{\rate{\tilde{k}}{R}{-}}{\rate{k}{P}{-}}\right)^{-1}
        \right)^{-1}.
\end{align}
So far this is exact, but using the assumption that
$\rate{k}{P}{-} \gg \rate{\tilde{k}}{R}{-}$, we have
\begin{align}
    \rate{k}{eff}{-}
    = \rate{\tilde{k}}{R}{-}
        \left( 1 + \frac{\rate{\tilde{k}}{R}{-}}{\rate{k}{P}{-}}\right)^{-1}
    \approx \rate{\tilde{k}}{R}{-}
        \left( 1 - \frac{\rate{\tilde{k}}{R}{-}}{\rate{k}{P}{-}}\right)
    \approx \rate{\tilde{k}}{R}{-}.
\end{align}

Now condsider $\rate{k}{eff}{+}$,
the effective on rate from the empty state to the doubly bound state.
This is more subtle because now the slow step is second rather than first.
The intuition is that $\rate{k}{eff}{+}$ should be approximately
$\rate{\tilde{k}}{R}{+}$, multiplied by a reduction factor which accounts for
the higher occupation probability of the empty state relative to
its actual origin point (the RNAP bound state).
If the separation of timescales is strong,
this ratio is approximately $\rate{k}{P}{+}/\rate{k}{P}{-}$,
so $\rate{k}{eff}{+} \approx
    \rate{\tilde{k}}{R}{+}\times\rate{k}{P}{+}/\rate{k}{P}{-}$.
\mmnote{I can come back and do this more carefully at some point,
but this is at the same level of accuracy as saying
$\rate{k}{eff}{-} \approx \rate{\tilde{k}}{R}{-}$.
Doing both at least preserves detailed balance so means are unaffected.}

Put another way, what this construction has done is preserve
the mean waiting time for transitions between the empty and doubly bound states.
But now the transition is modeled as a one step process,
rather than the two step it was in the four state picture of \fref{fig:4state}.
Note that this does \textit{not} leave $p_{RP}$ exactly unchanged
\mmnote{as I initially thought it did!},
only approximately.
The fractional change in occupation probability of the remaining three states
is of order $\rate{k}{P}{+}/\rate{k}{P}{-}$,
which is small compared to 1 to the same degree the weak promoter approximation is true.
To see this, note that when all on rates are exactly diffusion limited,
and in the absence energy consumption breaking detailed balance,
the relative weights in the four state picture would be
\begin{align}
\begin{split}
    p_E &\propto 1\\
    p_R &\propto \rate{k}{R}{+}/\rate{k}{R}{-}\\
    p_P &\propto \rate{k}{P}{+}/\rate{k}{P}{-}\\
    p_R &\propto \omega \left(\rate{k}{P}{+}/\rate{k}{P}{-}\right)
                        \left(\rate{k}{R}{+}/\rate{k}{R}{-}\right),
\end{split}
\label{eq:4state_eq_weights_rates}
\end{align}
where $\omega$ is a cooperativity factor.
In other words, we exactly recover the equilibrium picture.
The inclusion of transcription initiation and
imperfectly diffusion-limited on rates will partially break this simple form,
but it will remain approximately correct.

The same argument leading to \eref{eq:4state_eq_weights_rates}
applies in the three state model as well, but with $p_P$ removed
and the normalization updated to $p_E + p_R + p_{RP}$.
Note that $p_P < p_{RP}$ need not hold,
merely $p_P \ll 1$, for the approximation to be good.
But this is merely the weak promoter approximation,
so we are not actually assuming anything new or strong.

What our construction does \textit{not} guarantee is that the variance
(and higher moments) of the mRNA distribution will be unaffected.
We designed the effective rates so that the mean waiting time would be preserved.
But the waiting time distribution in the coarse-grained three state model
is exponential, as it must be for a memoryless Markov model.
If the separation of timescales between XapR and RNAP rates is sufficiently strong,
then the true two-step waiting time distribution will be approximately exponential.
But in the limit where the rates are equal, it would become gamma distributed!
This could impact the accuracy of predictions of Fano factor, variance,
and any other properties of the mRNA distributions higher moments beyond the mean.
How strong a timescale separation is sufficient will only become clear with numerics,
as analytical results beyond the mean in the four state model are nigh impossible.
\mmnote{Since Griffin's second round of experiments seem to confirm that
no gene expression is detectable in the absence of XapR,
I think these numerics are probably the top priority for theory now.
If my intuition about surgically removing the RNAP bound state is wrong,
we'll need a plan B soon to connect with the experiments.}

\mmnote{2019/01/28: Ned Wingreen had interesting thoughts.
He was overall optimistic about the whole project.
His main concern is what mechanisms could wreck our simple picture of
equilibrium or simple master equation for promoter occupancy:
\begin{enumerate}
\item is xanthosine toxic at high levels? Is it processed by any other pathways? How can we tell?
\item are there other things that could wreck our simple act construct? unknown binding sites?
Here we're hopefully ok b/c downstream is just yfp, and sort-seq will tell us about upstream.
But maybe there's something farther upstream out of range of sort-seq?
\item could XapR form tetramers in presence of DNA?
even if we don't see tetramers in absence of DNA?
could we tell somehow?
If the dimer-dimer affinity is weaker than the dimer-DNA affinity, is it moot?
Even fairly weak transient interactions can stabilize things a lot,
e.g., a dimer bound to each site could then interact and help prevent each other falling off DNA.
\item can we get higher expression somehow in the simple act construct?
3-fold induction over autofluorescence (if I read Griffin's prelim data right)
is not a lot of dynamic range to play with.
\item what if supercoiling is a problem? what would that look like experimentally?
sounds hard to disentangle in context of our simple model.
OTOH if expression levels are low maybe we needn't worry?
\end{enumerate}}

\appendix

\section{Reference for a simple-minded theorist with poor memory}

\subsection{Strains}
We have available five strains that are potentially useful/interesting:

1) $\Delta$\textit{XapA,B,R} (all knocked out).
Recall \textit{XapR} is just a bit downstream from \textit{XapA} \& \textit{B}.

2) Same as 1), but with wild type promoter (of \textit{XapA,B}) driving YFP.

3) wild type system, plus, at a separate locus, wt promoter driving YFP.

4) $\Delta$\textit{XapA,B,R}, YFP controlled by wt promoter,
\textit{XapR-mCh} under control of TetR (our standard dilution circuit)

5) like 4), except upstream binding site in wt promoter is knocked out.
This is the simple activation test vehicle.

%%%%%%%%%%%%%%%%%%%%% APPENDICES %%%%%%%%%%%%%%%%%%%%%%%%%%%%%%%%%%%
\appendix


\bibliographystyle{nature}
% \bibliographystyle{abbrv}
\bibliography{library}

\end{document}
